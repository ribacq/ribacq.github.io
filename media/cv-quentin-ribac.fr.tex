\documentclass[a4paper, 10pt, sans]{moderncv}

%% ModernCV themes
\moderncvstyle{classic} %casual, classic, oldstyle, banking
\moderncvcolor{blue} %blue, orange, green, red, purple, grey, black
\renewcommand{\familydefault}{\sfdefault}
\nopagenumbers{}

%% Character encoding
\usepackage[utf8]{inputenc}
\usepackage[T1]{fontenc}

%% French typography
\usepackage[french]{babel}

%% Adjust the page margins
\usepackage[scale=0.8, right=1cm, top=1cm, left=1cm, bottom=1cm]{geometry}
\recomputelengths

%% Personal data
\firstname{Quentin}
\familyname{Ribac}
\title{Développeur informaticien}
\quote{}

\address{21 boulevard Vauban}{51470 Saint Memmie}
\phone[mobile]{06~82~95~54~58}
\email{ribac.quentin@gmail.com}
\social[github]{ribacq}
\extrainfo{23 ans (né en octobre 1995)}
%\photo[64pt][0.4pt]{photo_cv.jpg} %facultative

%%------------------------------------------------------------------------------
%% Content
%%------------------------------------------------------------------------------
\begin{document}
\makecvtitle

\section{Compétences informatiques}
\cvdoubleitem{Langages}{C, C++, Java, Python, Go}{Conception}{UML}
\cvdoubleitem{Web}{HTML5, CSS3, PHP7, JavaScript (NodeJS, Meteor), PostgreSQL, httpd}{Documents}{\LaTeX, Libre Office}
\cvdoubleitem{GNU/Linux}{utilisé depuis 2010, Fedora, Bash, VIM, dwm}{Clavier}{Disposition BÉPO}

\section{Projets et Évènements}
\cventry{2018}{ODG games}{IUT de Lannion}{numérisation de jeux de société}{Projet Scrum, base SQL, serveur PHP, sockets NodeJS, JavaFX \& UI web, équipe de 6}{}
\cvitem{2018}{\textbf{sta}~: jeu textuel multi-joueur écrit en Go, projet personnel}
\cventry{2017, Janvier – Mai}{X Facteur}{}{implémentation du problème du voyageur de commerce dans un projet JavaFX}{}{IUT de Lannion}
\cventry{2017, Mars}{Portfolio}{IUT de Lannion}{Un simple site web statique écrit avec HTML, CSS et JavaScript (JQuery), et Markdown pour le formattage}{}{\url{https://ribacq.github.io}}
\cventry{2017, 19 et 20 mai}{Hackathon}{24h des DUT Informatique, Calais}{24h découpées en 8h IA, 8h web, 8h sécurité, en équipe de 6}{classé 9e sur 30}{}
\cventry{2017, 25 et 26 mars}{Hackathon}{Enssat, Lannion}{Conception et développement d'une application Android en 24h, en équipe de 5, GPS \& chat}{classé 3e sur 10}{}

\section{Formation}
\cventry{2016 – 2018}{DUT Informatique}{120 ECTS}{C, Conception orientée objet (Java), Gestion de projet, Économie et gestion d'entreprise, Anglais, Communication}{}{à Lannion (Côtes d'Armor)}
\cventry{2013 – 2016}{Cycle préparatoire aux études d'ingénieur}{138 ECTS}{Université de Technologie de Compiègne}{Informatique, anglais, mathématiques, mécanique, propriété intellectuelle}{à Compiègne (Oise)}
%\cventry{2013}{Baccalauréat Scientifique-SVT}{Mention Bien}{Spécialité Informatique et Sciences du Numérique, option europpéenne anglais}{}{à Châlons-en-Champagne (Marne)}

\section{Expérience professionnelle}
\cventry{2015, 3 semaines}{Intérim – suivi qualité}{Ecolab Production France SAS}{Suivi du remplissage de documents par les opérateurs, saisie informatique et compte-rendu aux responsables}{à Châlons-en-Champagne (Marne)}{}
\cventry{2014 \& 2016, 7 et 3 semaines}{Intérim – conditionnement}{Ecolab Production France SAS}{Conditionnement, préparation de poste}{à Châlons-en-Champagne (Marne)}{}

\section{Langues étrangères}
\subsection{Anglais (niveau européen C1)}
\cvitem{TOEIC}{Obtenu avec 990 points sur 990, mars 2018}
\cvitem{Écrit}{Compréhension technique et littéraire, conversation courante, rédaction de documents techniques}
\cvitem{Parlé}{Conversation courante et professionnelle, capacité à assister ou donner des exposés techniques ou visionner des films}

\section{Centres d'intérêt}
\cvitem{Littérature}{Traduction française complète de \emph{When Marnie Was There}, roman de Joan Robinson ; projet personnel}
\cvitem{Linguistique}{apprentissage autodidacte de notions de grammaire, de syntaxe et d'orthographe, \emph{constructed languages}}
\cvitem{C}{outils et jeux développés avec \texttt{ncurses.h}}
\cvitem{Lectures}{J.R.R. Tolkien, M. Proust, T. Pratchett, J. Gott}
%\cvitem{Chant}{chant choral (basse) depuis 1 ans et demi}

\end{document}

