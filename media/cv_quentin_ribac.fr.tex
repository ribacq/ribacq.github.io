\documentclass[a4paper, 10pt, sans]{moderncv}

%% ModernCV themes
\moderncvstyle{classic} %casual, classic, oldstyle, banking
\moderncvcolor{orange} %blue, orange, green, red, purple, grey, black
\renewcommand{\familydefault}{\sfdefault}
\nopagenumbers{}

%% Character encoding
\usepackage[utf8]{inputenc}
\usepackage[T1]{fontenc}

%% French typography
\usepackage[french]{babel}

%% Adjust the page margins
\usepackage[scale=0.8]{geometry}

%% Personal data
\firstname{Quentin}
\familyname{Ribac}
\title{Recherche de stage en développement logiciel — Bac+2}
\quote{}

\address{6a venelle des 3 avocats}{22300 Lannion}
\phone[mobile]{06~82~95~54~58}
\email{ribac.quentin@gmail.com}
\social[github]{https://github.com/ribacq}
\extrainfo{21 ans}
%\photo[64pt][0.4pt]{photo_cv.jpg} %facultative

%%------------------------------------------------------------------------------
%% Content
%%------------------------------------------------------------------------------
\begin{document}
\makecvtitle

\section{Compétences}
\cvdoubleitem{Langages}{C, C++, Java, Python}{Conception}{UML}
\cvdoubleitem{Web}{HTML5, CSS3, PHP5, SQL, httpd}{Documents}{\LaTeX, Libre Office}
\cvdoubleitem{GNU/Linux}{Fedora, Bash, VI, mount, dwm}{Clavier}{Disposition BÉPO}

\section{Expériences}
\cventry{2015, 3 semaines}{Intérim – suivi qualité}{Ecolab Production France SAS}{Suivi du remplissage de documents par les opérateurs, saisie informatique et compte-rendu aux responsables}{}{à Châlons-en-Champagne (Marne)}
\cventry{2014 \& 2016, 7 et 3 semaines}{Intérim – conditionnement}{Ecolab Production France SAS}{Conditionnement, préparation de poste}{}{à Châlons-en-Champagne (Marne)}

\section{Formation}
\cventry{actuellement}{Diplôme Universitaire de Technologie}{1\textsuperscript{ère} année}{C, Conception orientée objet (Java), Gestion de projet, Économie et gestion d’entreprise, Anglais, Communication}{}{à Lannion (Côtes d’Armor)}
\cventry{2013 – 2016}{Cycle préparatoire aux études d’ingénieur}{138 ECTS}{Université de Technologie de Compiègne}{Informatique, anglais, mathématiques, mécanique, propriété intellectuelle}{à Compiègne (Oise)}
\cventry{2013}{Baccalauréat Scientifique-SVT}{Mention Bien}{Spécialité Informatique et Sciences du Numérique, option europpéenne anglais}{}{à Châlons-en-Champagne (Marne)}

\section{Projets}
\cventry{2014 – 2016}{Traduction littéraire}{When Marnie Was There (Joan Robinson)}{Traduit de l’anglais vers le français}{}{Projet personnel}

\section{Anglais}
%faire un double item avec C/E O/E
\cvitem{Anglais}{C1, niveau professionnel oral et écrit, pratique régulière de l’anglais technique}

\section{Centres d’intérêt}
\cvitem{Linguistique}{apprentissage autodidacte de notions de grammaire, de syntaxe et d’orthographe, création d’une langue imaginaire}
\cvdoubleitem{GNU/Linux}{depuis 2010}{C}{outils et jeux avec \texttt{ncurses.h}}
\cvdoubleitem{Lecture}{J.R.R. Tolkien, M. Proust, T. Pratchett, J. Gott}{Chant}{chant choral (basse) depuis 1 ans et demi}

\end{document}

